ID2216 Developing Mobile Applications

Opposition report, group 4 WeCall

Rafael Aldana (rafaelap@kth.se)
Vincent Delitz (delitz@kth.se)
Ruth Eriksson (ruthe@kth.se)

Introduction
WeCall is an application that wants to make people aware of neglected contacts and help to find small windows of opportunity to call and nurture important social connections.
User experience
When you open the application you get a good overview of your friends, with pictures, in sections. Each section is a time frame. This makes it easy to see the contacts you have neglected the most. Also there is a suggestion on top of the page. Each section is scrollable, which is not quite obvious at first. When you click/tap a contact you can see name, phone number, last time called, address and local time. 
There is also a button for a map feature for the address. The local time feature is very useful when you have a lot of friends and family abroad. The application is simple and easy to use yet it has some nice extra features like the map and the time zone information.

Technical implementation
The application was built with Android Studio and Eclipse and based on existing code. It is great that you were able to find code to repurpose and extend rather than writing all code from scratch. It is good practice doing so and hopefully you saved some time and debugging. 
Did you benefit from using two different development tools? Was Eclipse useful for your purpose?
The application reads phonebook information and references contact data including call logs using manifest.permission class. This seems to be working with no problem!
From main view users can open detail view of a contact and view address, phone number, and time zone. Users can initiate a call and click/tap a symbol which launches location based services offered by Google. The three implemented interfaces, Google Maps Android, Geocoding and Time Zone seem to work well in the application and look like they are well integrated. 

Design choices
Regarding main view, have you thought about making the user able to call a person direct from main view with for example a long click. That would eliminate the need of loading detail view and the user would save one click/tap and a wait for a load. The scrollable sections are not obvious/self-explanatory, maybe a symbol or an arrow that shows the user that you can scroll down or up.
The detail view has information about the contact but the content and the tags melt together because they are the same font and size, maybe make one of them bold or larger.
Over all great work and idea. Good luck developing!
